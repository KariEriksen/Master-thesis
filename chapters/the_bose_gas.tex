OBS: change the wording
The Schrödinger equation can in principle describe systems containing any number of particles as long as they are moving within the non relativistic range of velocity, i.e. not moving closer to the speed of light. This is the case of the Bose-Einstein condensate and one of the reasons for the amount it is been explored. 
In the next chapter we will look closer at some boson systems like the Bose-Einstein condensate and its characteristics, and how we can solve these type of systems with an analytic approach and a numerical one. 
Maybe specify some constants if that is necessary. 

\section{Ideal Bose gas}
Bosons are quantum particles with integer spin, meaning they follow Bose-Einstein statistics. 
Elementary: Higgs boson, photon, gluon, gravition

Composite: hydrogen, mesons, nucleus of deuterium

Consider a dilute Bose gas where the inter-particle distance is much greater than the scattering length. It can be approximated by a mean field theory. We describe it by a simple hard-sphere effective potential. (Master Joachim)

Gross-Pitaevskii - analogy to the Schrödinger equation. 

Ground state energy

The ideal bose gas is straight forward and have been solved analytically and numerically by many. Weak interacting bosons is also a case that has been studied to a great extent. If we where to add strong interaction the system may do a number of things, flipping of spin, excitation etc.


\section{The Bose-Einstein Condensation}
The ability of bosons at low temperature to collectively inhabit the same energy state, compared to fermions who are restricted to the Pauli exclusion principle.

Free Bose gas - 

\begin{equation} \label{eq:hamilt_free_bose_gas}
H^{free}_V = \int_V dx \frac{1}{2m} \nabla a*(x). \nabla a(x)
\end{equation}

BEC in interaction - 

BEC in trap - 

The Bose-Einstein condensation ground state can be described by the Gross-Pitaevskii equation, which uses the Hartree-Fock approximation and the pseduopotential intereaction model. (wiki) 
A a many-body boson system, the Gross-Pitaevskii approach is a mean-field approach. (MBBS)
If the spacing between the particles are greater than the scattering length (in the so-called dilute limit), then one can approximate the true interaction potential that features in this equation by a psedopotential. READ more about why this is.

The wave function in the Hartree-Fock approximation of a system of N bosons is represented by the product of single-particle wave functions $\psi$

\begin{equation}
\psi(r_1, r_2, \cdots, r_N) = \psi(r_1) \psi(r_2) \cdots \psi(r_N)
\end{equation}

The pseudopotential model Hamiltonian of the system is given as 

\begin{equation}
H = \sum_{i=1}^N \left( -\frac{\hbar^2}{2m} \frac{\partial^2}{\partial r_i^2} + V(r_i) \right) + \sum_{i<j} \frac{4 \pi \hbar^2 a_s}{m} \delta (r_i - r_j)
\end{equation}

m is the mass of the boson, V is the external potential, $a_s$ is the boson-boson scattering length and $\delta(r)$ the Dirac delta-function. This is an effective potential, the original  with modifications 

The Gross-Pitaevskii equation reads, 

\begin{equation}
\left( -\frac{\hbar^2}{2m} \frac{\partial^2}{\partial r^2} + V(r) + \frac{4 \pi \hbar^2 a_s}{m}|\psi(r)|^2 \right) \psi(r) = \mu \psi (r)
\end{equation}

Se 104 MBBS

Solutions: free particle V(r) = 0

Soliton?

Thomas-Fermi approximation

Bogoliubov approximation

Interacting Bose gas - Landau's phenomenological theory of superfluidity. Based on the idea that a quantum liquid remains a classical fluid even at zero temperature  and that the classical hydrodynamical laws remain valid. (Many body boson system.) 
Many models have been suggested to create a microscopical theory of the superfluidity. Vortex ring model, hard-sphere model, guassian cluster approach. (wiki)

The properties of these liquids are described in terms of the spectrum of the collective excitations.

\section{The wave function}
We are studying a system of two electrons confined in a harmonic oscillator trap described by the Hamiltonian 

\begin{equation}\label{eq:hamilt}
\hat{H} = \sum_{i=1}^N \left( - \frac{1}{2} \nabla_i^2 + \frac{1}{2} \omega^2 r_i^2 \right) + \sum_{i<j} \frac{1}{r_{ij}}
\end{equation}

where the first sum is the standard harmonic oscillator part and the last is the interacting part between the electrons and $N$ represent the number of particles. $\omega$ is the oscillator frequency of the trap and $r_i$ is the position of particle $i$, whereas $r_{ij}$ is the distance between the particles and given as $r_{ij} = |\mathbf{r_i} - \mathbf{r_j}|$. \\

\begin{equation} \label{eq:trap_potential}
 V_{ext}(\mathbf{r}) = 
 \Bigg\{
 \begin{array}{ll}
	 \frac{1}{2}m\omega_{ho}^2r^2 & (S)\\
 \strut
	 \frac{1}{2}m[\omega_{ho}^2(x^2+y^2) + \omega_z^2z^2] & (E)
 \end{array}
\end{equation}

\begin{equation} \label{eq:potential_internal}
 V_{int}(|\mathbf{r}_i-\mathbf{r}_j|) =  \Bigg\{
 \begin{array}{ll}
	 \infty & {|\mathbf{r}_i-\mathbf{r}_j|} \leq {a}\\
	 0 & {|\mathbf{r}_i-_r\mathbf{r}_j|} > {a}
 \end{array}
\end{equation}

\begin{equation} \label{eq:trialwf}
 \Psi_T(\mathbf{r})=\Psi_T(\mathbf{r}_1, \mathbf{r}_2, \dots \mathbf{r}_N,\alpha,\beta)=\prod_i g(\alpha,\beta,\mathbf{r}_i)\prod_{i < j}f(a,|\mathbf{r}_i-\mathbf{r}_j|),
\end{equation}

\begin{equation} \label{eq:spf_g}
g(\alpha,\beta,\mathbf{r}_i)= \exp{[-\alpha(x_i^2+y_i^2+\beta z_i^2)]}.
\end{equation}

\begin{equation} \label{eq:jastrow_f}
    f(a,|\mathbf{r}_i-\mathbf{r}_j|)=\Bigg\{
 \begin{array}{ll}
	 0 & {|\mathbf{r}_i-\mathbf{r}_j|} \leq {a}\\
	 (1-\frac{a}{|\mathbf{r}_i-\mathbf{r}_j|}) & {|\mathbf{r}_i-\mathbf{r}_j|} > {a}.
 \end{array}
\end{equation}

Our wave function is given from the energy of the restricted Boltzmann machine, see \eqref{eq:E_rbm}, which is the joint energy functional between the visible and hidden nodes. From the marginal probability of the joint probability distribution, see \eqref{eq:F_rbm_marg}, we get our wave equation.                     

\begin{align}\label{eq:F_rbm}
\Psi(X) &= F_{rbm}(X) \\
&= \frac{1}{Z} \exp \left( -\sum_{i}^{M} \frac{(X_i - a_i)^2}{2 \sigma^2} \right) \prod_{j}^{N} \left( 1 + \exp \left( b_j + \sum_{i}^{M} \frac{X_i \omega_{ij}}{\sigma^2} \right) \right)
\end{align}

Here $Z$ is the partition function, $X_i$ represents the visible nodes running up to $M$, and $a_i$ and $b_j$ are the biases described in the section below, \eqref{sec:rbm}, where number of hidden nodes $j$ runs up to $N$. 
$\omega_{ij}$ is an $M \times N$ matrix holding the weights connecting the visible nodes with the hidden and $\sigma$ is the standard deviation of the noise in our model. \\

\section{Interaction}