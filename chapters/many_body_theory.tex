\section{Canonical quantization}

Moving from a classical Hamiltonian description to a non-relativistic quantum mechanical description of the particle through a procedure called canonical quantization.(ref. Simen Kvaal)

Canonical communtation relations 

\begin{equation} \label{eq:commutation}
\{q_k, p_l\} = \delta_{kl}
\end{equation}

Any coordinate system $\epsilon = (q,p)$ we wish to use that satisfies \ref{eq:commutation} can be called canonical. (ref. Simen Kvaal)

Classical state: point in phase space

Quantum state: complex-valued wave function in an infinite dimensional Hilbert space, i.e. a complete vector space with an inner product.
Standard choice of Hilbert space is the space of square integrable function, $L^2$, where the wave function $\psi$ is a function depending on all coordinates:

\begin{equation}
\psi = \psi (x_1, x_2, \cdots, x_N)
\end{equation}

where $(x_1, x_2, \cdots , x_N) \in X^N$ is a point in the configuration space of N particles. 

The wavefunction is a map 

\begin{equation}
\psi : X^N \longrightarrow \mathbb{C}
\end{equation}

An observable is a $\textit{self-adjoint operator}$ (Hermitian operator) $\hat{\Omega}$, and its value in the state $\psi$ is the expectation value. 

Expectation value: 

\begin{equation}
\expval{\Omega} = \expval{\hat{\Omega}}{\psi} = \int \psi (x_1, \cdots , x_N)\ast [\hat{\Omega} \psi (x_1, \cdots, x_N)] dx_1 \cdots dx_N
\end{equation}

\begin{equation}
i \hbar \frac{\partial}{\partial t} \ket{\psi(t)} = \hat{H} \ket{\psi(t)}
\end{equation}

Replace classically canonical coordinates $\epsilon = (q, p)$ with operators $(\hat{q}, \hat{p})$ such that

\begin{equation}
\{q_j, p_k\} = \delta_{jk} \Rightarrow [\hat{q_j}, \hat{p_k}] = i \hbar \delta_{jk}
\end{equation}

\section{Second quantization}

Now the particles themselves are discrete quanta created and destroyed with creation and annihilation operators

Position operator 

\begin{equation}
[\hat{r_i} \psi](x_1, \cdots, x_N) = r_i \psi (x_1, \cdots, x_N) 
\end{equation}

Momentum operator 

\begin{equation}
[\hat{p_i} \psi](x_1, \cdots, x_N) = -i \hbar \nabla_i \psi (x_1, \cdots, x_N) 
\end{equation}

Fock space

\section{The Schrödinger equation}

The many-body Hamiltonian (s. 12) 

one-body operator  

two-body operator

\section{Wave function}