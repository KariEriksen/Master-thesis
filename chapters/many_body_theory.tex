Some introduction containing historical information and stuff, and motivation for this section. 
Configuration interaction, Many-body perturbation theory, Coupled-cluster 
Electron correlation,

Solving the Schrödinger equation, analytical and now numerical

Many-body algebraic methods originate from quantum field theory \cite{shavitt2009many}

\section{Classical Mechanics}

In classical mechanics a particles state is described by two variables, $x(t)$ and $p(t)$, the particles position and its momenta. For a system of N particles the state would be formed by vectors holding each particles position and momentum, $q = (\vec{r}_1, \cdots,\vec{r})_N$ and $p = (\vec{p}_1, \cdots,\vec{p}_N)$. 
Together they form a point $\xi(q, p)$ in a two-dimensional \textit{phase space} $\in \mathbb{R}^{2 \cdot n}$ \cite{kvaal}. The phase space contains all the possible values for the position and momentum variables.
The state variables are governed by \textit{Hamilton's equation of motion},

\begin{align}
\dot{q} &= \frac{\partial}{\partial p} \mathscr{H} (q, p) \\
\dot{p} &= -\frac{\partial}{\partial q} \mathscr{H} (q, p),
\end{align}

where $\mathscr{H}$ is the $\textit{Hamiltonian}$, an operator that we interpret as the total energy of the system. To see this we write out a special case within the classical Hamiltonian dynamics

\begin{align}
\mathscr{H}(q,p) &= \mathscr{T}(p) + \mathscr{V}(q) + \mathscr{W}(q) \\
&= \frac{1}{2m} \sum^N_{i=1} |\vec{p}_i|^2 + \sum^N_{i=1} v(\vec{r}_i) + \frac{1}{2} \sum^N_{i \neq j} w(r_{ij})
\end{align}.

This is the Hamiltonian for N particles of mass m, with inter-particle reaction through the force of a central potential $w(r_{ij}) = |\vec{r}_i - \vec{r}_j|$, moving in an external potential field $v(\vec{r})$. We recognize the first term as the $\textit{kinetic energy}$, the second term as the $\textit{external potential energy}$, and the final term as the $\textit{interaction energy}$.

\section{Quantum Mechanics}

In ordinary quantum mechanics an observable is a linear operator acting on a Hilbert space. Position operator and momentum operator, other observable quantities like angular momentum, energy, and so on, are linear operators constructed out of linear combinations of products of the position and momentum.\cite{kvaal}
Atomic units, all equal 1.
Hamiltonian 

\subsection{Canonical quantization}

Moving from a classical Hamiltonian description to a non-relativistic quantum mechanical description of the particle through a procedure called canonical quantization.(ref. Simen Kvaal)

Canonical communtation relations 

\begin{equation} \label{eq:commutation}
\{q_k, p_l\} = \delta_{kl}
\end{equation}

Any coordinate system $\epsilon = (q,p)$ we wish to use that satisfies \ref{eq:commutation} can be called canonical. (ref. Simen Kvaal)

Classical state: point in phase space

Quantum state: complex-valued wave function in an infinite dimensional Hilbert space, i.e. a complete vector space with an inner product.
Standard choice of Hilbert space is the space of square integrable function, $L^2$, where the wave function $\psi$ is a function depending on all coordinates:

\begin{equation}
\psi = \psi (x_1, x_2, \cdots, x_N)
\end{equation}

where $(x_1, x_2, \cdots , x_N) \in X^N$ is a point in the configuration space of N particles. 

The wave function is a map 

\begin{equation}
\psi : X^N \longrightarrow \mathbb{C}
\end{equation}

An observable is a $\textit{self-adjoint operator}$ (Hermitian operator) $\hat{\Omega}$, and its value in the state $\psi$ is the expectation value. 

Expectation value: 

\begin{equation}
\expval{\Omega} = \expval{\hat{\Omega}}{\psi} = \int \psi (x_1, \cdots , x_N)\ast [\hat{\Omega} \psi (x_1, \cdots, x_N)] dx_1 \cdots dx_N
\end{equation}

\begin{equation}
i \hbar \frac{\partial}{\partial t} \ket{\psi(t)} = \hat{H} \ket{\psi(t)}
\end{equation}

Replace classically canonical coordinates $\epsilon = (q, p)$ with operators $(\hat{q}, \hat{p})$ such that

\begin{equation}
\{q_j, p_k\} = \delta_{jk} \Rightarrow [\hat{q_j}, \hat{p_k}] = i \hbar \delta_{jk}
\end{equation}

\subsection{Canonical transformation}

A one-to-one algebra morphism of an observable, u, onto itself leaving the canonical commutation relations invariant. The canonical transformation is a map, $u \rightarrow u$ 

properties satisfying 

-linearity
-invariance
-product conserving (s. 175 MBBS)

\subsection{Second quantization}

Now the particles themselves are discrete quanta created and destroyed with creation and annihilation operators

Position operator 

\begin{equation}
[\hat{r_i} \psi](x_1, \cdots, x_N) = r_i \psi (x_1, \cdots, x_N) 
\end{equation}

Momentum operator 

\begin{equation}
[\hat{p_i} \psi](x_1, \cdots, x_N) = -i \hbar \nabla_i \psi (x_1, \cdots, x_N) 
\end{equation}

Fock space

\section{The Schrödinger equation}

The many-body Hamiltonian (s. 12) 

one-body operator  

two-body operator

\section{The Harmonic Oscillator}



\section{Wave function}